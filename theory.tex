\documentclass[11pt]{article}

\usepackage{amsmath, amssymb, bm}

% Margins
\topmargin=-0.45in
\evensidemargin=0in
\oddsidemargin=0in
\textwidth=6.5in
\textheight=9.0in
\headsep=0.25in

\title{DSPotential Theory Document}
\author{ David Moeller Sztajnbok }
\date{March, 2024}

\begin{document}
\maketitle

%--Paper--

\section{Brief Vector Calculus Review}
Before diving into the differential equations that govern fluid flow, particularly the flow we are interested in - irrotational, incompressible flow - a brief review of vector calculus is useful for the math that is to follow. This document will be void of complicated derivations, but will still include basic vector calculus concepts and notation. Below is a brief review of these concepts. \\
\subsection{Total Derivative}
\noindent
In single-variable calculus, functions are dependent on one variable only. Therefore, some function $f(x)$ has as its derivative $\frac{df}{dx}$ and will be a function of x and x only (or some constant.)\\ \\
\noindent
In multivariable calculus, however, functions are dependent on more than one variable. It is typically the case in fluid flow that quantities are functions of space ($x, y,$ and $z$) as well as time ($t$) which can be typically written as $f(x, y, z, t)$. However, the one of the variables in the function can be itself a function of another variable. It is imperative in that case to make use of the chain rule when differentiating the function, just as was the case in single-variable calculus when differentiating compound functions. \\ \\
\noindent
Let us consider a practical example. The acceleration of a fluid particle is given by:

\begin{equation*}
    \vec{\bm{a}} = \frac{d}{dt}[\vec{\bm{u}}(\vec{\bm{r}}, t)]
\end{equation*}

\noindent
Where $\vec{\bm{r}}$ is the position vector given by:

\begin{equation*}
    \vec{\bm{r}} = \begin{bmatrix}
                    r_x \\
                    r_y \\
                    r_z \\     
                    \end{bmatrix}
\end{equation*}

\noindent
And the components of the velocity vector $\vec{\bm{u}}$ are given by:

\begin{equation*}
    \vec{\bm{u}} = \begin{bmatrix}
                    u \\
                    v \\
                    w \\     
                    \end{bmatrix}
\end{equation*}

\noindent
The total derivative $\frac{d}{dt}[\vec{\bm{u}}(\vec{\bm{r}}, t)]$ must account for changes in position as functions of time, as well as changes in the velocity components as functions of position and time. The chain rule must be applied The total derivative is from now on denoted $\frac{D}{Dt}$ and is given by:

\begin{equation*}
    \frac{D\vec{\bm{u}}}{Dt} = \frac{\partial\vec{\bm{u}}}{\partial t} + \frac{\partial\vec{\bm{u}}}{\partial x}\frac{\partial x}{\partial t} + \frac{\partial\vec{\bm{u}}}{\partial y}\frac{\partial y}{\partial t} + \frac{\partial\vec{\bm{u}}}{\partial z}\frac{\partial z}{\partial t}
\end{equation*}

\noindent
Note that in the equation above, $r_x, r_y$ and $r_z$ are simply $x, y$ and $z$ since we will have the velocity not as a function of the particle's position (Lagrangian approach to fluid flow analysis) but rather as a function of the space coordinates themselves (Eulerian approach to fluid flow analysis.) \\ \\
\noindent
The expression can be simplified. Note that $\frac{\partial x}{\partial t}$, $\frac{\partial y}{\partial t}$, and $\frac{\partial z}{\partial t}$ are, respectively, the $x, y, $ and $z$ components of the velocity vector, i.e., $u, v,$ and $w$. Additionally, recall the gradient operator $\nabla$:

\begin{equation*}
    \nabla = \begin{bmatrix}
                \frac{\partial}{\partial x} \\
                \frac{\partial}{\partial y} \\
                \frac{\partial}{\partial z} \\     
                \end{bmatrix}
\end{equation*}

\noindent
Such that:

\begin{equation*}
    \nabla \vec{\bm{u}} = \begin{bmatrix}
                            \frac{\partial\vec{\bm{u}}}{\partial x} \\
                            \frac{\partial\vec{\bm{u}}}{\partial y} \\
                            \frac{\partial\vec{\bm{u}}}{\partial z} \\
                            \end{bmatrix}
\end{equation*}

\noindent
And the dot product of the velocity vector itself with the gradient of the velocity vector is given by:

\begin{equation*}
    \vec{\bm{u}} \cdot \nabla \vec{\bm{u}} = u\frac{\partial\vec{\bm{u}}}{\partial x} + v\frac{\partial\vec{\bm{u}}}{\partial y} + w\frac{\partial\vec{\bm{u}}}{\partial z}
\end{equation*}

\noindent
Note, therefore, that the total derivative can be simplified to:

\begin{equation*}
    \frac{D\vec{\bm{u}}}{Dt} = \frac{\partial\vec{\bm{u}}}{\partial t} + \vec{\bm{u}} \cdot \nabla \vec{\bm{u}}
\end{equation*}

\noindent
In fact, the total derivative of any flow field quantity can be written as:

\begin{equation*}
    \frac{D(\blacksquare)}{Dt} = \frac{\partial(\blacksquare)}{\partial t} + \vec{\bm{u}} \cdot \nabla (\blacksquare)
\end{equation*}

\noindent
And it applies to any variable, be it scalar-valued or vector-valued.\\

\subsection{Curl and Divergence}
\noindent
Another important set of concepts important in vector calculus that involve the gradient operator, $\nabla$, are the curl and divergence of a vector field. The curl of a vector field $\vec{\bm{A}}$ is given by:

\begin{equation*}
    \textnormal{curl}(\vec{\bm{A}}) = \nabla \times \vec{\bm{A}}
\end{equation*}

\noindent
While the divergence of a vector field $\vec{\bm{A}}$ is given by:

\begin{equation*}
    \textnormal{div}(\vec{\bm{A}}) = \nabla \cdot \vec{\bm{A}}
\end{equation*}

\noindent
This will show itself useful when we define the concepts of rotationality, as well as the fundamental condition for the conservation of mass. \\

\pagebreak
\section{Differential Form of Conservation Laws}
Lorem Ipsum \\

%--/Paper--

\end{document}