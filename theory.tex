\documentclass[11pt]{article}

\usepackage{amsmath, amssymb, bm}

% Margins
\topmargin=-0.45in
\evensidemargin=0in
\oddsidemargin=0in
\textwidth=6.5in
\textheight=9.0in
\headsep=0.25in

\title{DSPotential Theory Document}
\author{ David Moeller Sztajnbok }
\date{March, 2024}

\begin{document}
\maketitle

%--Paper--

\section{Brief Vector Calculus Review}
Before diving into the differential equations that govern fluid flow, particularly the flow we are interested in - irrotational, incompressible flow - a brief review of vector calculus is useful for the math that is to follow. This document will be void of complicated derivations, but will still include basic vector calculus concepts and notation. Below is a brief review of these concepts. \\
\subsection{Vector Notation}
This document will use standardized notation throughout, some of which is presented below.\\ \\
\noindent
Vectors are denoted by an arrow above them and are also bolded. A velocity vector, for instance, is denoted $\vec{\bm{u}}$ while a scalar quantity, density, is denoted $\rho$.\\ \\
\noindent
Vectors are presented in matrix form to more easily visualize operations like dot products, cross products, and vector multiplication. They are presented as an $nx1$ matrix, that is, a column vector. The example of velocity vector is shown below:

\begin{equation*}
    \vec{\bm{u}} = \begin{bmatrix}
                    u \\
                    v \\
                    w \\
                    \end{bmatrix}
\end{equation*}

\noindent
Where $u, v, $ and $w$ are the $x, y, $ and $z$ components of vector $\vec{\bm{u}}$. \\

\subsection{Vector Operations}
It is important to remember the different vector operations. Addition and subtraction are ommited from the discussion below. The focus is given to the scalar, vectorial, and simple multiplication of vectors, also known as the dot, cross, and simple products. 
\subsubsection{The Dot Product}
Consider two vectors $\vec{\bm{a}}$ and $\vec{\bm{b}}$ such that:

\begin{equation*}
    \vec{\bm{a}} = \begin{bmatrix}
        a_x \\
        a_y \\
        a_z \\
    \end{bmatrix}
\end{equation*}

And

\begin{equation*}
    \vec{\bm{b}} = \begin{bmatrix}
        b_x \\
        b_y \\
        b_z \\
    \end{bmatrix}
\end{equation*}

\noindent
Then the dot product between them is:

\begin{equation*}
    \vec{\bm{a}} \cdot \vec{\bm{b}} = a_xb_x + a_yb_y + a_zb_z
\end{equation*}

\noindent
With the geometric meaning that the dot product is the magnitude of the projection of $\vec{\bm{a}}$ onto $\vec{\bm{b}}$, also given by:

\begin{equation*}
    \vec{\bm{a}} \cdot \vec{\bm{b}} = \bar{\vec{\bm{a}}}\bar{\vec{\bm{b}}}\cos{\theta}
\end{equation*}

\begin{equation*}
    \vec{\bm{a}} \cdot \vec{\bm{b}} = \begin{vmatrix}
        \vec{\bm{a}}
    \end{vmatrix}\begin{vmatrix}
        \vec{\bm{b}}
    \end{vmatrix}
    \cos{\theta}
\end{equation*}

\noindent
Where $\theta$ is the angle formed between $\vec{\bm{a}}$ and $\vec{\bm{b}}$.\\ \\ \\
\noindent
\emph{Note: the dot product is an operation between two vectors and it results in a scalar. It is not to be confused with the multiplication of a scalar by a vector:}
\begin{equation*}
    c\vec{\bm{a}} = \begin{bmatrix}
        ca_x \\
        ca_y \\
        ca_z \\
    \end{bmatrix}
\end{equation*}
\emph{Where $c$ is a constant. The operation above is the product between a scalar and a vector, and results in a vector! So be weary of the dots in the text below.}

\subsubsection{The Cross Product}
Consider the same vectors as before. The cross product between them is:

\begin{equation*}
    \vec{\bm{a}} \times \vec{\bm{b}} = \begin{vmatrix}
        \hat{i} & \hat{j} & \hat{k} \\
        a_x & a_y & a_z \\
        b_x & b_y & b_z
    \end{vmatrix} = \hat{i}\begin{vmatrix}
        a_y&a_z\\b_y&b_z
    \end{vmatrix} - \hat{j}\begin{vmatrix}
        a_x&a_z\\b_x&b_z
    \end{vmatrix} + \hat{k}\begin{vmatrix}
        a_x&a_y\\b_x&b_y
    \end{vmatrix} = \hat{i}(a_yb_z - a_zb_y) + \hat{j}(a_zb_x-a_xb_z) + \hat{k}(a_xb_y - a_yb_x)
\end{equation*}\\

\noindent
With the geometric meaning that the cross product is a vector normal to both $\vec{\bm{a}}$ and $\vec{\bm{b}}$ and its magnitude is the area formed by the parallellepiped between $\vec{\bm{a}}$ and $\vec{\bm{b}}$. This magnitude is:

\begin{equation*}
    \vec{\bm{a}} \times \vec{\bm{b}} = \begin{vmatrix}
        \vec{\bm{a}}
    \end{vmatrix}\begin{vmatrix}
        \vec{\bm{b}}
    \end{vmatrix}
    \sin{\theta}
\end{equation*}\\

\noindent
Where $\theta$ is the angle between the vectors. Note that the cross product is an operation between two vectors that yields another vector.

\subsection{Scalar vs. Vector Fields}
It is important to make a subtle yet important distinction before speaking of derivatives. In fluid flow, we are interested in calculating different properties at different points in space and time. Therefore, quantities will be functions of space ($x, y, \textnormal{and}\ z$) and time ($t$.) \emph{However, these quantities can be scalar and/or vectors.}\\\\
\noindent
A vector field is a function of space and time wherein each point in space and time defines a vector. The best example and perhaps the most relevant here is the velocity vector, $\vec{\bm{u}}$:

\begin{equation*}
    \vec{\bm{u}} = u\hat{i} + v\hat{j} + w\hat{k}
\end{equation*}\\
\noindent
Wherein each component of the velocity is a function of space and time:


\begin{eqnarray*}
    u = u(x, y, z, t)\\
    v = v(x, y, z, t)\\
    w = w(x, y, z, t)
\end{eqnarray*}\\
\noindent
Therefore at each point in space at a given time, each component of the velocity vector at that point in space and time is defined, and a vector is therefore defined at that point.\\ \\
\noindent
Scalar fields, on the other hand, are scalars defined as functions of space and time. Thus, instead of a point in space and time defining three components that form a vector, only a scalar is defined. The best examples are pressure and temperature:
\begin{eqnarray*}
    P = P(x, y, z, t)\\
    T = T(x, y, z, t)\\
\end{eqnarray*}

\subsection{The Gradient of a Scalar Field}
Consider a scalar field $p(x,y,z)$. It defines a pressure (scalar) at every point in space. We are often interested in the direction at which the pressure increases the most, and the rate at which it increases. This is given by the gradient of the scalar field, $\nabla p$. \\ \\
\noindent
The gradient of a scalar field $\nabla p$ is defined as the vector with the following properties:
\begin{enumerate}
    \item Its direction is towards the greatest increase in the scalar field and;
    \item Its magnitude is the rate of increase of the scalar field in that direction.
\end{enumerate}
\noindent
The gradient vector, $\nabla$ is defined as:
\begin{equation*}
    \nabla = \begin{bmatrix}
                \frac{\partial}{\partial x} \\
                \frac{\partial}{\partial y} \\
                \frac{\partial}{\partial z} \\     
                \end{bmatrix}
\end{equation*}
\noindent
Such that the gradient of the scalar field is simply:
\begin{equation*}
    \nabla p = \begin{bmatrix}
                \frac{\partial p}{\partial x} \\
                \frac{\partial p}{\partial y} \\
                \frac{\partial p}{\partial z} \\     
                \end{bmatrix}
\end{equation*}

\subsection{Curl and Divergence}
\noindent
Another important set of concepts important in vector calculus that involve the gradient operator, $\nabla$, are the curl and divergence of a vector field. The curl of a vector field $\vec{\bm{A}}$ is given by:

\begin{equation*}
    \textnormal{curl}(\vec{\bm{A}}) = \nabla \times \vec{\bm{A}}
\end{equation*}

\noindent
While the divergence of a vector field $\vec{\bm{A}}$ is given by:

\begin{equation*}
    \textnormal{div}(\vec{\bm{A}}) = \nabla \cdot \vec{\bm{A}}
\end{equation*}

\noindent
This will show itself useful when we define the concepts of rotationality, as well as the fundamental condition for the conservation of mass. \\


\subsection{Total Derivative}
\noindent
In single-variable calculus, functions are dependent on one variable only. Therefore, some function $f(x)$ has as its derivative $\frac{df}{dx}$ and will be a function of x and x only (or some constant.)\\ \\
\noindent
In multivariable calculus, however, functions are dependent on more than one variable. It is typically the case in fluid flow that quantities are functions of space ($x, y,$ and $z$) as well as time ($t$) which can be typically written as $f(x, y, z, t)$. However, the one of the variables in the function can be itself a function of another variable. It is imperative in that case to make use of the chain rule when differentiating the function, just as was the case in single-variable calculus when differentiating compound functions. \\ \\
\noindent
Let us consider a practical example. The acceleration of a fluid particle is given by:

\begin{equation*}
    \vec{\bm{a}} = \frac{d}{dt}[\vec{\bm{u}}(\vec{\bm{r}}, t)]
\end{equation*}

\noindent
Where $\vec{\bm{r}}$ is the position vector given by:

\begin{equation*}
    \vec{\bm{r}} = \begin{bmatrix}
                    r_x \\
                    r_y \\
                    r_z \\     
                    \end{bmatrix}
\end{equation*}

\noindent
And the components of the velocity vector $\vec{\bm{u}}$ are given by:

\begin{equation*}
    \vec{\bm{u}} = \begin{bmatrix}
                    u \\
                    v \\
                    w \\     
                    \end{bmatrix}
\end{equation*}

\noindent
The total derivative $\frac{d}{dt}[\vec{\bm{u}}(\vec{\bm{r}}, t)]$ must account for changes in position as functions of time, as well as changes in the velocity components as functions of position and time. The chain rule must be applied The total derivative is from now on denoted $\frac{D}{Dt}$ and is given by:

\begin{equation*}
    \frac{D\vec{\bm{u}}}{Dt} = \frac{\partial\vec{\bm{u}}}{\partial t} + \frac{\partial\vec{\bm{u}}}{\partial x}\frac{\partial x}{\partial t} + \frac{\partial\vec{\bm{u}}}{\partial y}\frac{\partial y}{\partial t} + \frac{\partial\vec{\bm{u}}}{\partial z}\frac{\partial z}{\partial t}
\end{equation*}

\noindent
Note that in the equation above, $r_x, r_y$ and $r_z$ are simply $x, y$ and $z$ since we will have the velocity not as a function of the particle's position (Lagrangian approach to fluid flow analysis) but rather as a function of the space coordinates themselves (Eulerian approach to fluid flow analysis.) \\ \\
\noindent
The expression can be simplified. Note that $\frac{\partial x}{\partial t}$, $\frac{\partial y}{\partial t}$, and $\frac{\partial z}{\partial t}$ are, respectively, the $x, y, $ and $z$ components of the velocity vector, i.e., $u, v,$ and $w$. Additionally, recall the gradient operator $\nabla$:

\begin{equation*}
    \nabla = \begin{bmatrix}
                \frac{\partial}{\partial x} \\
                \frac{\partial}{\partial y} \\
                \frac{\partial}{\partial z} \\     
                \end{bmatrix}
\end{equation*}

\noindent
Such that:

\begin{equation*}
    \vec{\bm{u}} \cdot \nabla = u \frac{\partial}{\partial x} + v \frac{\partial}{\partial y} + w \frac{\partial}{\partial z}
\end{equation*}

\noindent
And the above times the velocity vector itself is:

\begin{equation*}
    (\vec{\bm{u}} \cdot \nabla)\vec{\bm{u}} = u \frac{\partial\vec{\bm{u}}}{\partial x} + v \frac{\partial\vec{\bm{u}}}{\partial y} + w \frac{\partial\vec{\bm{u}}}{\partial z} 
\end{equation*}

\noindent
Note, therefore, that the total derivative can be simplified to:

\begin{equation*}
    \frac{D\vec{\bm{u}}}{Dt} = \frac{\partial\vec{\bm{u}}}{\partial t} + (\vec{\bm{u}} \cdot \nabla)\vec{\bm{u}}
\end{equation*}

\noindent
In fact, the total derivative of any flow field quantity can be written as:

\begin{equation*}
    \frac{D\phi}{Dt} = \frac{\partial\phi}{\partial t} + (\vec{\bm{u}} \cdot \nabla)\phi
\end{equation*}

\noindent
And it applies to any variable, be it scalar-valued or vector-valued.\\

\pagebreak
\section{Differential Form of Conservation Laws}
Lorem Ipsum \\

%--/Paper--

\end{document}